% !TEX root = BA-Bauer

\newpage
\section{Gegenstand und Ziel der Arbeit}

Gegenstand und Ziel der Arbeit ist die Etwicklung eines für den Endbenutzer ausgelegten Gerätes zur Aufnahme und Wiedergabe von DMX-Daten. Die Entwicklung umfasst das Erstellen einer entsprechenden elektrischen Schaltung, das Planen und Herstellen einer zugehörigen Platine, sowie das Bestücken dieser. Die Platine soll in einem passenden Gehäuse eingebaut werden. Außerdem wird eine Software entwickelt, die zum einen die Aufnahme und Wiedergabe der DMX-Daten ermöglicht und zum Anderen eine Benutzerschnittstelle bildet mit der der Benutzer das Gerät möglichst intuitv bedienen kann. Grundlegende Funktionsprinzipien sind bereits in der vorangegangenen Bachelorpraxisprojektarbeit aufgezeigt worden \cite{Bauer2021}, welche in dieser Arbeit weiterentwickelt und erweitert werden. \\Folgende Haupt- und Nebenanforderungen sind für die Entwicklung definiert. \\

\hspace*{-5mm}\textbf{Hauptanforderungen:}\\
- Möglichkeit mehrere Aufnahmen anzulegen\\
- Einstellbare Aufnahmezeit\\
- Aufname eines DMX-Stroms und -Standbildern\\
- Benutzeroberfläche mit Display und Menüführung\\
- Maßgeschneidertes Platinendesign\\
- Gehäuse\\
- Handliche Größe\\

\hspace*{-5mm}\textbf{Nebenanforderungen:}\\
- Intuitive Benutzbarkeit\\
- Optisch Ansprechendes Design des Gehäuses\\
- Optimierung des Speicherbedarfs auf der SD-Karte\\
- Möglichkeit die Wiedergabegeschwindigkeit während der aktiven Wiedergabe zu ändern\\


% !TEX root = BA-Bauer

\newpage
\section{Einleitung}
Einmal pro Jahr wird die Nacht der Kultur in der Göttinger Innenstadt veranstaltet, ein vielfältiges Stadtfest der Kulturszene. Das Göttinger Wahrzeichen, das Gänseliesel, soll eindrucksvoll mit bewegtem und statischem Licht in Szene gesetzt werden. Für den Lichttechniker bedeutet diese Aufgabe die Installation und Verkabelung aller Lampen, die Programmierung einer Lichtshow und die Betreuung dieser während der ganzen Nacht. Die Lichtshow wird von einem Laptop mit DMX-Interface oder von einem Lichtsteuerpult aus gesteuert. Der Diebstahl eines dieser Geräte würde einen großen finanziellen Verlust bedeuten. Mit dem Gerät, welches in dieser Arbeit entwickelt wird können die Daten der vorab programmierten Lichtshow aufgezeichnet werden. Am Tag der Veranstaltung muss der Lichttechniker nur noch die Lampen installieren, verkabeln und die gespeicherte Lichtshow von dem Gerät abspielen. Damit bleibt ihm Zeit, die Veranstaltung als Gast zu genießen.\\
Die in der vorangegangenen Praxisprojektarbeit erarbeiteten Grundlagen zur Aufnahme und Wiedergabe von DMX-Daten haben die technische Umsetzbarkeit aufgezeigt. Der entwickelte Prototyp ist in seiner Funktionalität jedoch sehr eingeschränkt und ist den Anforderungen eines realen Anwendungsfalls nicht gewachsen. Ziel dieser Arbeit ist die Weiterentwicklung und Erweiterung der erarbeiteten Inhalte, sowie die Implementierung in einen Prototypen des marktfähigen Produktes.\\
%Umstand eigener Bedarf: um ein pers Problem zu lösen, war es notwendig sich mit der Materie auseinanderzusetzen. das resultierte in großem Interesse für das Kernthema der BA
%Für die Entwicklung werden hauptsächlich Kenntnisse der Elektrotechnik, besonders aus dem Bereich der Schaltungstechnik und Informatik verwendet. 
%Motivation: pers. Interesse 
%Schaltungselektronik, Platinendesign, Informatik, CAD, 3D-Druck, 
Zu Beginn wird das Grundprinzip der benötigten elektrischen Schaltung erläutert. Daraufhin werden die für die Umsetzung des Prinzips benötigten Hardwarekomponenten und deren Beschaltung näher erläutert, sowie deren Funktion in der Gesamtschaltung aufgezeigt. Anschließend wird gezeigt, wie die abstrakte Schaltung in ein reales Platinendesign überführt wird. Die Platine wird in einem Gehäuse montiert, dessen Konstruktion und Design dargestellt wird. Nach der Behandlung der Hardware werden die wichtigsten Funktionen der Software betrachtet und in den Gesamtkontext gestellt. Dafür wird zunächst auf die grundlegende Softwarearchitektur eingegangen und die verwendeten Hilfsmittel, die zur Konfiguration der Hardware und Entwicklung der Software verwendet werden, werden vorgestellt. Die wichtigsten Funktionen, die zur Verarbeitung der Eingaben des Benutzers benötigt werden, und die Funktionsweise der Benutzer-Rückmeldungen für die einzelnen Komponenten werden erläutert. Die überarbeitete Aufnahme- und Wiedergabefunktion, sowie die neu hinzugekommenen Modi werden außerdem erläutert. Zuletzt wird auf den Aufbau und die Umsetzung des Menüs und der entsprechenden Menüführung, sowie auf die Einstellungsmöglichkeiten eingegangen.
%Paar Funktionen nennen.
% !TEX root = BA-Bauer.tex
\subsection{LCD}
Für die Programmierung des LCD-Displays wird ein HAL-Treiber (HD44780-Stm32HAL) von Olivier Van den Eede verwendet und erweitert. Die Treiberdateien befinden sich im Anhang \ref{CD-Anhang}. Der Treiber ist für die Steuerung von HD44780-Punktmatrixtreibern für LCD-Displays und ursprünglich für die Verwendung mit den Mikrokontrollern der STM32 F1-Serie entwickelt. Auf dem in dieser Arbeit verwendeten LCD-Modul befindet sich ein SPLC780-Treiberchip zum steuern des Displays und ist kompatibel mit den Anweisungen für HD44780-Treiberchips. Der Treiber kann LCD-Displays der Größe 16xN und 20xN Zeichen steuern. Standardmäßig ist die Größe 16x2 aktiviert, kann jedoch durch auskommentieren der entsprechenden Konstanten-Definition verändert werden. 
\begin{lstlisting}[firstnumber=16, language=C, caption = lcd.h: Einstellung Displaygröße, label = code:lcdsize]
	#define LCD20xN 		// For 20xN LCDs
	//#define LCD16xN			// For 16xN LCDs
\end{lstlisting}
Damit der Treiber mit den Mikrokontrollern der STM32 F4-Serie kompatibel ist, muss zudem die includierte Datei $stm32f1xx\_hal.h$ mit $stm32f4xx\_hal.h$ in der Header-Datei ersetzt werden.
\newline
Um das LCD-Display verwenden zu können benötigt der Treiber Informationen über die Verbindungen der Pins des LCD-Moduls mit den Pins des MCUs. Die Verbindungsinformationen werden für die Datenleitungen $D0$ bis $D7$ in zwei Arrays ($ports$ und $pins$), jeweils eins für Pin und Port, global gespeichert. Der Zellenindex (0-7) der Arrays entspricht der jeweiligen Datenleitung ($D0$-$D7$). Zusätzlich zu den Informationen zu den Datenleitungen werden die Informationen für den $RS$ und $EN$-Pin benötigt. Diese werden zusammen mit den Zeigern auf die Arrays $pins$ und $ports$ der Funktion $lcd\_create(...)$ (Codeausschnitt \ref{code:lcdcreate}) übergeben. Außerdem wird der 8-Bit Modus des Treibers durch die Übergabe des Prameters $LCD\_8\_BIT\_MODE$ aktiviert um schnellere Schreibraten zu erreichen, wie bereits in Kapitel \ref{sec:HardLCD} erläutert.
\begin{lstlisting}[firstnumber=302, language=C, caption = main.c: Funktionsaufruf lcd\_create(...), label = code:lcdcreate, breaklines = true]
lcd = Lcd_create(ports, pins, LCD_RS_GPIO_Port, LCD_RS_Pin, LCD_E_GPIO_Port, LCD_E_Pin, LCD_8_BIT_MODE);
\end{lstlisting}

Soll ein Zeichen an einer bestimmten Stelle des LCD-Displays gezeigt werden, so muss zunächst der sogenannte $cursor$ an der entsprechenden Stelle platziert werden. Für diesen Zweck ist die Funktion \textit{Lcd\_cursor(Lcd\_HandeTypeDef $*$ lcd, uint8\_t row, uint8\_t col)} im Treiber zu finden. Mit dem Parameter $row$ wird die Zeile und mit dem Parameter $col$ die Spalte der Position des Cursors festgelegt, wobei der Wert 0 der ersten Zeile oder Spalte entspricht. Der Cursor verschiebt sich um eine Stelle nach rechts nachdem ein neues Zeichen dargestellt ist. Ist das Ende einer Zeile erreicht, wird der Cursor an den Anfang der nächsten Zeile verschoben. Diese Funktionsweise ermöglicht es einzelne bereits dargestelle Zeichen zu überschreiben, ohne den Inhalt des gesamten LCD-Displays erneut beschreiben zu müssen.
\newline
Bei dem Beschreiben des Display sind des öfteren Fehler aufgetreten, bei dem zusätzliche, willkürliche Zeichen nach dem Ende der auszugebenden Zeichenkette ausgegeben werden. Der Grund für diesen Fehler ist die Unfähigkeit der Ausgabefunktion das Ende der Zeichenkette zu erkennen. Aus diesem Grund wird für die Ausgabe die zusätzliche Funktion $void\ Lcd\_string\_length(...)$ in Codeausschnitt \ref{code:stringlength} dem Treiber hinzugefügt, bei dessen Aufruf die Anzahl der auszugebenden Zeichen als Parameter übergeben wird. Damit kann sichergegangen werden, dass nur eine bestimmte Anzahl an Zeichen ausgegeben werden.
\begin{lstlisting}[firstnumber=95, language=C, caption = lcd.c: Funktion Lcd\_string\_length(...), label = code:stringlength]
void Lcd_string_length(Lcd_HandleTypeDef * lcd, char * string, uint8_t length)
{
	for(uint8_t i = 0; i < length; i++)
	{
		lcd_write_data(lcd, string[i]);
	}
}
\end{lstlisting}
Umgesetzt wird diese Funktion mit einer einfachen $for$-Schleife, dessen Anzahl an Wiederholungen durch den Übergabeparameter $uint8\_t\ length$ bestimmt werden. Mithilfe der Funktion $lcd\_write\_data(..)$ werden einzelne Bytes in das Datenregister des SPLC780-Chips geschrieben und auf dem LCD-Display ausgegeben.
Zum Treiber ist außerdem eine Funktion hinzugekommen, mit der eine gesamte Zeile des LCD-Displays zurückgesetzt werden kann. Die enstrpechende Zeile wird der Funktion übergeben und mithilfe einer $for$-Schleife jedes Zeichen der Zeile mit einem Leerzeichen überschrieben.\\
\newline
\textbf{Optimierungsmöglichkeiten}\\
Grundsätzlich bietet der Treiber eine stabile Grundlage für einfache Ausgaben auf dem Zeichen-LCD-Display, jedoch nur wenn die Funktion $lcd\_string\_length(...)$ mit der entsprechenden Anzahl an Zeichen der auszugebenden Zeichenkette aufgerufen wird. Dafür muss während der Programmierung die Anzahl der Zeichen in der Zeichenkette händisch bestimmt und der Funktion übergeben werden und bietet damit ein großes Fehlerpotential. Besser wäre eine zuverlässige und automatische Erkennung der Zeichenkettenlänge in der Funktion $lcd\_string(...)$.
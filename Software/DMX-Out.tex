% !TEX root = BA-Bauer.tex
\subsection{Ausgabe eines DMX-Datenpaketes}
Die Funktion zum Senden von DMX-Datenpakten wurde im Vergleich zu der Praxisarbeit optimiert und vereinfacht. Der Vollständige Sendevorgang wird mit nur zwei Funktionen ausgeführt. Codeausschnitt \ref{code:dmx.c-Transmit} zeigt die Funktion zum Senden eines DMX-Datenpaketes, welche als einzige aufgerufen werden muss um ein Datenpaket zu senden.
\lstinputlisting[
caption = DMX.c: Transmit-Funktion,
label = code:dmx.c-Transmit, 
language = C, 
firstnumber = 66, 
firstline = 66, 
lastline = 79]
{/Users/Felix/Documents/CubeMX/BPA-Code/Core/Src/DMX.c}
Die Übergabeparameter der Funktion sind eine DMX-Struktur ($struct$\footnote{text}) $DMX_TypeDef*\ hdmx$ und die Anzahl der zu übertragenden Datenbytes \textit{uint16\_t size}. In der DMX-Struktur befinden sich Informationen über die zu verwendende UART-Schnittstelle, %Macht nur Sinn wenn das im Code auch benutzt wird
sowie das Array $TxBuffer$ mit den zu übertragenen Datenbytes. Zu beginn der Funktion wird in Zeile 68 der Treiberchip über das Beschreiben des Pins $DMX\_DE\_Pin$ mit einem High-Pegel aktiviert. Daraufhin wird der Ausgangspin der UART-Schnittstelle mithilfe der Funktion $DMX\_set\_TX\_Pin\_manual()$ manuell beschreibbar gemacht und mit einem High-Pegel beschrieben. Das Prinzip dieses Vorgang ist in der Praxisarbeit beschrieben \cite{Bauer2021}. %Seite einfügen
Das Zählerregister ($CNT$) des Timers 11 (\textit{htim11}) wird mit dem Wert 0 überschrieben und das $UIF$-Bit \textit{Update Interrupt Flag} im Statusregister des Timers zurückgesetzt. Dieses Bit wird hardwareseitig gesetzt wenn der Timer übergelaufen ist. Mit der Funktion $SET\_BIT(...)$ wird das $CEN$ Bit gesetzt, wodurch der Timer startet. Der Timer ist zuvor in CubeMX als $one-pulse$-Timer%stimmt das??
konfiguriert. Sobald der Timer übergelaufen ist, wird er gestoppt und zählt den Zähler $CNT$ nicht weiter hoch. Mit der $while$-Abfrage in Zeile 74 wird auf den Überluf des Timers gewartet. Ist er übergelaufen, so wird das $UIF$-Bit wieder zurückgesetzt und der Ausgangspin in Zeile 76 mit einem Low-Pegel beschrieben. Zu diesem Zeitpunkt ist das DMX-Startsignal vollständig ausgegeben und die Datenbytes können gesendet werden. Um der UART-Schnittstelle wieder Zugriff auf den Ausgangspin zu gewähren wird mit der Funktion \textit{DMX\_set\_TX\_Pin\_auto()} die manuelle Beschreibbarkeit des Pins deaktiviert. Anschließend wird mit dem Funktionsaufruf in Zeile 78 die Übertragung der Daten gestartet. Bei dieser Funktion handelt es sich um eine nicht-blockierende Funktion. Die Daten werden Interruptbasiert gesendet. Sind alle Daten vollständig gesendet, so wird durch einen Interrupt der UART-Schnittstelle die Funktion in Codeausschnitt \ref{code:dmx.c-txIT} aufgerufen.
\lstinputlisting[
caption = DMX.c: Interrupt-Funktion ausgehender UART-Daten,
label = code:dmx.c-txIT, 
language = C, 
firstnumber = 120, 
firstline = 120, 
lastline = 128]
{/Users/Felix/Documents/CubeMX/BPA-Code/Core/Src/DMX.c}
Der Inhalt der Funktion soll nur ausgeführt werden, wenn ein aktiver Sendevorgang ausgeführt wird, also wenn die Variable \textit{Univers.sending} den Wert 1 enthält. Ist dies der Fall, so wird der UART-Ausgangspin ein weiteres Mal mit der Funktion \textit{DMX\_set\_TX\_Pin\_manual} manuell beschreibbar gemacht und der Pin mit einem Low-Pegel beschrieben um das $Brake$-Signal des DMX-Protokolls zu senden. Anschließend wird der Zustand der entsprechenden LED invertiert um den Benutzer eine Rückmeldung über einen erfolgreichen Sendevorgang eines Datenpaketes zu geben.\\
\textbf{Optimierungsmöglichkeiten}\\
Wie auch in der Praxisarbeit wird das Signal ohne $Interbytedelay$, also Verzögerungszeit zwischen den einzelnen Datenbytes, gesendet. Dadurch wird das eingehende SIgnal nicht originalgetreu wiedergegeben, jedoch antsteht daraus kein Datenverlust oder eine Veränderung der Wiedergabefrequenz der Datenpakete. Außerdem wird in der aktuellen Version des Programmcodes das $Interbytedelay$ während der Aufnahme nicht gemessen oder registriert. Um diese Funktionalität bei der Wiedergabe zu implementieren muss zunächst eine entsprechende Messung des $Interbytedelays$ während der Aufnahme erfolgen.
% !TEX root = BA-Bauer.tex

\subsection{Encoder}
Für den Programcode des Encoders kommt ein ähnliches Prinzip wie für die Taster im vorigen Kapitel zum Einsatz. Der Aufruf der entsrechenden Funktion erfolgt über einen Interrupt.
Wie bereits in Kapitel \ref{sec:Encoder} beschrieben, muss der Zustand beider Terminals des Encoders ausgelesen werden um eine Drehrichtung bestimmen zu können. Terminal $A$ ist in CubeMX als externer Interrupteingang mit Erkennung einer fallenden Flanke, Terminal $B$ als Standard-Eingang konfiguriert. Wird durch eine fallende Flanke des Signals des Terminals $A$ ein Interrupt ausgelöst, so entscheidet der Zustand des Terminals $B$ über die Drehrichtung des Encoders. In der Interrupt-Funktion muss also lediglich der Zustand von Terminal $B$ geprüft werden. 
\lstinputlisting[
caption = Encoder.c: Interrupt-Funktion,
label = code:enc.c, 
language = C, 
firstnumber = 45, 
firstline = 45, 
lastline = 60]
{/Users/Felix/Documents/CubeMX/BPA-Code/Core/Src/Encoder.c}
Codeausschnitt \ref{code:enc.c} zeigt den Inhalt der Interruptfunktion. Zunächst wird in Zeile 47-49, wie auch in der Interrupt-Funktion der Taster, eine softwareseitige Entprellug durchgeführt. Die Verzögerungszeit fällt für den Encoder mit 50\,ms deutlich kürzer aus als für die Taster, um auch schnelle Drehbewegungen registrieren zu können. Eine längere Verzögerungszeit hat einen Verlust von getätigten Eingaben, besonders bei höheren Drehgeschwindigkeiten, zur Folge. 50\,ms haben sich in Tests als angenehm zu bedienen erwiesen und einen gute Balance zwischen maximaler Eingabegeschwindigkeit und Zuverlässigkeit der Eingaben hergestellt.
Nach der Entprellung folgt die Abfrage des Zustands des Terminals $B$ mithilfe der Funktion $HAL\_GPIO\_ReadPin(..)$. Ist der Zustand $High$, so wird die global verfügbare Variable $enc\_position$ um 1 dekrementiert wenn sie größer als 0 ist. Ist der Zustand $Low$, so wird die Variable um 1 inkrementiert. Bei der Variable $enc\_position$ handelt es sich um eine 16-Bit Variable ohne Vorzeichen, welche $2^{16} = 65536$ Werte umfasst und damit genug Platz für Eingaben bietet.\\
\textbf{Optimierungsmöglichkeiten}\\
Bei der Entwicklung des Schaltplans ist ein Fehler im Bereich der CR-Filter zum Entprellen des Encoders nach der Fertigung der Platine aufgefallen. Durch diesen Fehler wird die Beschaltung des Encoders unwirksam im Bezug auf dessen Entprellung. Mit der Behebung könnte die sofwareseitige Entprellung reduziert werden oder sogar vollständig entfallen. Dadurch kann die Responsivität des Encoders verbessert werden.
Außerdem wäre eine Erkennung der Drehgeschwindigkeit sinnvoll um bei einer größeren Geschwindigkeit die Inkrement- und Dekrementgröße zu variieren, wodurch schnellere und präzise Eingaben gleichermaßen möglich sind.
% !TEX root = BA-Bauer.tex

\subsection{DMX-Struktur}
\label{sec:dmx_struct}
Informationen die für die Aufnahme oder Wiedergabe der DMX-Daten benötigt werden, werden an verschiedenen Stellen des Programms eingeholt und an anderen Stellen wiederrum benötigt. Um die Informationen einfach zugänglich zu machen, werden diese mithilfe einer Struktur gebündelt. Abbildung \ref{code:dmx-struct} zeigt die Definition dieser Struktur, welche von der Praxisprojektarbeit übernommen und erweitert ist. Die Struktur ähnelt einer Klassendefinition in C++. Durch die Deklaration einer Instanz der Struktur zum Beispiel mit dem Namen $Univers$, können die Variablen der Struktur über die Instanz erreicht werden. Die Variable $rec\_time$ würde über $Univers.rec\_time$ erreichbar sein und kann gelesen und beschrieben werden. Damit die Instanz an allen Stellen im Programm verwendet werden kann, wird sie global definiert. Die in den folgenden Kapitel verwendeten Variablen werden den in dieser Struktur befindlichen zugewiesen oder die Werte aus ihnen gelesen.
\lstinputlisting[
caption = DMX.h: Typdefinition DMX\_TypeDef,
label = code:dmx-struct, 
language = C, 
firstnumber = 24, 
firstline = 24, 
lastline = 39]
{/Users/Felix/Documents/CubeMX/BPA-Code/Core/Inc/DMX.h}
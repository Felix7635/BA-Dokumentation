% !TEX root=BA-Bauer.tex

\subsection{Taster}
Um eine einwandfreie Funktion der Taster zu garantieren, ist nicht nur deren Schaltung, sondern auch ein entsprechender Programmcode wichtig. Im folgenden Kapitel wird auf den Programmcode, der zum auslesen eines Tasterdrucks verwendet wird, eingegangen. Ein Tasterdruck kann zu jedem Zeitpunkt des Programmablaufs geschehen und sollte möglichst vom Programmcode nicht unbeachtet bleiben. Um diese Funktionalität zu gewährleisten, sind die Taster mit interruptfähigen Eingängen der MCU verbunden. Die Beschaltung des Tasters mit einem MCU internen Pull-Up-Widerstand sorgt dafür, dass an den Tastern im nicht betätigten Zustand ein High-Pegel und während der Betätigung ein Low-Pegel anliegt. Der Interrupt muss also auslösen, wenn sich der Logikpegel eines Tasters von einem High- auf einen Low-Pegel ändert, also eine fallende Flanke des Taster-Signals erkannt wird. In CubeMX sind die Taster-Pins so konfiguriert, dass eine Interrupt-Routine beim Auftreten einer fallenden Flanke aufgerufen wird. Diese Routine ruft eine sogenannte Callback-Funktion auf, in der die Betätigung des Tasters bearbeitet wird. Code-Ausschnitt \ref{code:btn.h} zeigt einen Ausschnit aus der Datei $button.h$. Jedem Taster wird ein Zahlenwert der Basis 2 zugewiesen. Diese spiegeln die Bits 1-4 eines Bytes wider.
\lstinputlisting[
caption = button.h: Definitionen und Funktionsprototypen,
label = code:btn.h, 
language = C, 
firstnumber = 5, 
firstline = 5, 
lastline = 12]
{/Users/Felix/Documents/CubeMX/BPA-Code/Core/Inc/button.h} 
Codeausschnitt \ref{code:btn.c} zeigt einen Teil der Callback-Funktion. Zunächst wird in Zeile 30 geprüft, ob in den letzten 200\,ms ($DELAYTIME$) bereits ein Aufruf der Funktion erfolgt ist. Diese Abfrage soll mehrfache Eingaben, ausgelöst durch ein mechanisches Schwingen (Prellen) des Tasters, verhindern. Falls sie bereits aufgerufen wurde, so wird die Funktion beendet, da der Aufruf wahrscheinlich durch das mechanische Schwingen des Taster aufgerufen wurde. Falls nicht wird die Variable $last\_updated$ mit der aktuellen Zeit aktualisiert. Daraufhin wird mithilfe einer $switch$-Anweisung festgestellt, welcher Taster den Interrupt ausgelöst hat. Dafür wird die Variable $GPIO\_Pin$, welche von der Interrupt-Funktion an die Callback-Funktion übergeben wird, mit den einzelnen Pins der Taster verglichen. 
\lstinputlisting[
caption = button.c: Codeausschnitt Taster-Interruptfunktion,
label = code:btn.c, 
language = C, 
firstnumber = 28, 
firstline = 28, 
lastline = 33]
{/Users/Felix/Documents/CubeMX/BPA-Code/Core/Src/button.c}
Ist der auslösende Pin festgestellt, wird der Variable $presses$ mithilfe einer bitweisen $oder$-Operation der entsprechende Zahlenwert des Tasters, der in der Datei $button.h$ (Codeausschnitt \ref{code:btn.h}) definiert ist, wie folgt zugewiesen.
\lstinputlisting[
caption = button.c: Bitweise Zuordnung eines betätigten Tasters,
label = code:bitwise_or, 
language = C, 
firstnumber = 37, 
firstline = 37, 
lastline = 41]
{/Users/Felix/Documents/CubeMX/BPA-Code/Core/Src/button.c}
Der Vorteil der bitweisen $oder$-Operation ist, dass nur das entsprechende Bit gesetzt wird ohne alle anderen zu verändern. Ergänzende Abfragen ob das Bit bereits gesetzt ist, entfallen.
Im Hauptprogramm werden ausschließlich die Funktionen $Button\_pressed(uint8\_t\ button)$ für die Abfrage, ob ein bestimmter Taster betätigt wurde, und $Button\_reset()$, um die Variable $presses$ zurückzusetzen, verwendet. Bei der Abfrage eines Tasters wird das entsprechende Bit wieder zurückgesetzt. Wurden zwei verschiedene Taster innerhalb kurzer Zeit betätigt, können beide nacheinander oder gleichzeitig mit der Funktion $Button\_pressed(...)$ abgefragt werden. Codeausschnitt \ref{code:btn_pressed} zeigt den Inhalt dieser Funktion. Zunächst folgt eine $if$-Abfrage, zum registrierten Zeitpunkt der letzten Tastereingabe. Ist die Eingabe nicht älter als der Wert von $TIMEOUT$, so wird zunächst die temporäre Variable $temp$ mit dem Wert 0 beschrieben. Mithilfe einer bitweisen $und$-Operation der durch die Callback-Funktion beschriebenen Variable $presses$ und dem Wert des abgefragten Tasters wird die Variable $temp$ mit dem entsprechenden Wert beschrieben. Zuletzt werden die abgefragten Bits aus der Variable $presses$ mit einer $0$ überschrieben. Der Rückgabewert der Funktion ist die Variable $temp$. Im Hauptprogramm kann anhand eines Rückgabewertes ungleich $0$ erkannt werden, dass der oder einer der abgefragten Taster betätigt wurde.
\lstinputlisting[
caption = button.c: Abfrage eines Tasters,
label = code:btn_pressed, 
language = C, 
firstnumber = 10, 
firstline = 10, 
lastline = 21]
{/Users/Felix/Documents/CubeMX/BPA-Code/Core/Src/button.c}
Zur Veranschaulichung des Ablaufs der Funktion soll im Folgenden als Beispiel abgefragt werden, ob der Zurück-Taster betätigt wurde. Im Hauptprogramm wird die Funktion \textit{Button\_pressed(BACK)} dazu aufgerufen. Die Konstante $BACK$ besitzt den Wert 1, also binär 0000 0001. Kurz bevor die Funktion aufgerufen wurde, wurden der Zurück- und Bestätigung-Taster betätigt. Der Inhalt der Variable $presses$ ist dadurch binär 0000 0011. Die bitweise $und$-Operation der beiden Variablen ergibt binär 0000 0001. Der Rückgabewert der Funktion ist somit 1. Damit kann im Hauptprogramm von einer Betätigung des Zurück-Taster ausgegangen werden. Wird nun allerdings die gleiche Funktion ein zweites Mal aufgerufen, ohne dass der Zurück-Taster betätigt wurde, so ist der Inhalt der Variable $presses$ binär 0000 0010. Das Bit des Zurück-Taster wurde mithilfe der Operation in Zeile 19 zurückgesetzt. Die $und$-Operation hat dann das Ergebnis 0000 0000. Der Rückgabewert der Funktion ist null, der abgefragte Taster ist nicht betätigt worden.
\newline
\textbf{Optimierungsmöglichkeiten}\\
Die Konstante $TIMEOUT$ besitzt keine Zuordnung zu einem einzelnen Taster. Wurde vor längerer Zeit ein Taster betätigt und kurz vor der Abfrage ein anderer Taster betätigt, so scheint die Betätigung des ersten Tasters auch innerhalb der Timeout-Zeit geschehen zu sein. Dieses Problem kann gelöst werden, indem eine Variable pro Taster angelegt wird, in der die Zeit der letzten Betätigung des jeweiligen Tasters gespeichert wird. Diese ersetzt die globale letzte Aktualisierungszeit $last\_updated$. Der Nachteil dieser Methode ist eine deutlich aufwendigere Abfrage in der Funktion $Button\_pressed(...)$ und Zuweisung in der Callback-Funktion. Eine zweite Möglichkeit das Problem zu lösen ist der Einsatz eines Timers. Dieser wird gestartet, wenn ein Taster betätigt wird. Läuft der Timer über, wird in der Interrupt-Funktion des Timers der Betätigungszustand in der Variable $presses$ für alle Taster zurückgesetzt. Soll nur der Zustand eines Tasters zurückgesetzt werden, so muss je ein Timer pro Taster zur Verfügung stehen, da in der Interrupt-Funktion des Timers nicht unterschieden werden kann, welcher Taster den Timer gestartet hat.
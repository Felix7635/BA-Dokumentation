% !TEX root = BA-Bauer.tex

\subsection{Wiedergabefunktion}
Für die Wiedergabe der aufgenommenen DMX-Daten werden zwei Wiedergabemodi benötigt. Die Aufnahmedaten der Standard-, Trigger- und Endlosaufnahme werden für die Wiedergabe gleich behandelt. Die Aufnahmedaten der $Step$-Aufnahme müssen gesondert behandelt werden, da diese keine Zeitinformation enthalten. Für beide Aufnahmemodi muss zunächst die entsprechende Aufnahmedatei ausgewählt werden.

\subsubsection{Auswahl der Aufnahmedatei}
\label{sec:selectfile}
Um dem Benutzer eine Auswahl der Aufnahmedatei zu ermöglichen, muss zunächst die SD-Karte nach vorhandenen Dateien durchsucht werden. Zudem müssen die Namen abgerufen und auf dem LCD-Display ausgegeben werden. Mithilfe der Funktion $f\_findfirst(..)$ kann die SD-Karte an einem bestimmten Dateipfad nach Dateien mit einem bestimmten Muster im Dateinamen durchsucht werden. Alle Aufnahmedateien werden mit der Endung $.dmx$ gespeichert. Das Suchmuster wird mit $*.dmx$ festgelegt und bedeutet, dass jede Datei, deren letzte vier Zeichen $.dmx$ beinhalten, gefunden werden. Der Funktion $f\_findfirst(...)$ wird zudem eine $FILINFO$ Struktur namens $info$ übergeben. Wird eine Datei passend zu dem festgelegten Suchmuster gefunden, so wird unter anderem der vollständige Dateiname in dem Zeichen-Array $info.fname$ gespeichert. Ist das Array leer, wurde keine Datei gefunden. Nachdem die erste Datei gefunden ist, müssen die restlichen auf der SD-Karte befindlichen Dateien gefunden werden. Dafür wird die Funktion $f_findnext(...)$ verwendet. Diese Funktion wird so lange aufgerufen, bis keine weitere Datei gefunden wird. Nach jedem erfolgreichen Aufruf wird der Dateiname in ein multidimensionales Zeichen-Array gespeichert, welches 20 Dateinamen beinhalten kann. Der Inhalt dieses Arrays wird dazu benutzt die Dateinamen auf dem LCD-Display darzustellen. Durch drehen des Encoders wird die Auswahl verändert. Ist eine Auswahl getroffen, wird sie mit dem Bestätigen-Taster bestätigt. Die Auswahl der Aufnahmedatei kann jederzeit durch Betätigen des Zurück-Taster abgebrochen werden.\\
\newline
\textbf{Optimierungsmöglichkeiten}\\
Die Auswahlfunktion ist durch die Größe des multidimensionalen Zeichen-Arrays auf die Anzeige von maximal 20 Dateinamen begrenzt. Für die Weiterentwicklung der Software ist es sinnvoll eine Lösung zu finden, bei der die Anzahl der maximal anzuzeigenden Dateien nicht begrenzt ist.

%Im Gegensatz zu den mehreren Aufnamefunktionen gibt es für die Wiedergabe der DMX-Daten nur eine Wiedergabefunktion. Aus den auf der SD-Karte während der Aufnahme gespeicherten Informationen in der $.nfo$-Datei kann bestimmt werden, wie die gespeicherten Daten wiedergegeben werden müssen. Zunächst startet die Funktion mit der Auswahl der wiederzugebenen Aufnahmedatei. Dafür wird die SD-Karte nach Dateien durchsucht die eine $.dmx$ Dateiendung besitzen. Die Dateinamen werden dann auf dem Display angezeigt und können durch das Drehen des Encoders ausgewählt werden. Mit dem Bestätigungs-Taster wird die Auswahl festgelegt und das erste Datenbyte gelesen. Auf dem Display erscheint eine Abfrage ob die Wiedergabe gestartet werden soll. Wird ein weiteres Mal der Bestätigungstaster betätigt, startet die Wiedergabe. Sie kann mit einer Betätigung des Zurück-Tasters gestoppt werden.

\subsubsection{Wiedergabe einer kontinuierlichen Aufnahme}
Nachdem eine Aufnahmedatei erfolgreich ausgewählt ist, werden die zugehörige Info-Datei geöffnet und die Daten gelesen. Ist die darin befindliche Aufnahmedauer größer als null, handelt es sich um eine kontinuierliche Aufnahme. Für die Wiedergabe wird der Standardmodus verwendet. Ein Flussdiagramm der Funktion befindet sich im Anhang \ref{fluss:playconti}. Als erstes wird ein Millisekunden Timer gestartet, woraufhin vier Bytes aus der Aufnahmedatei gelesen werden. Der Inhalt der ersten vier Bytes ist der bei der Aufnahme registrierte Zeitpunkt, an dem das DMX-Datenpaket vollständig empfangen ist, wie in Kapitel \ref{sec:save_data} erläutert ist. Bei der Wiedergabe fungiert diese Zeit als Startzeitpunkt des jeweiligen Datenpaketes. Anschließend werden einzelne Bytes aus der Datei gelesen, bis entweder das Ende der Datei erreicht ist, die maximale Anzahl DMX-Kanälen eines Datenpaketes erreicht ist oder das NPC-Zeichen erkannt wird. Ist das Ende der Datei erreicht, wird mithilfe der Funktion $f\_lseek(...)$ zum Beginn der Datei zurückgekehrt und der Zähler des gestarteten Timers auf 0 zurückgesetzt. Ist das NPC-Zeichen erkannt oder die maximale Anzahl an Bytes empfangen, so wird die am Anfang gelesene Startzeit des Datenpaketes mit dem Zähler des Timers verglichen, bis der Zähler den gleichen oder einen höheren Wert als die Startzeit besitzt. Erst danach wird das DMX-Datenpaket gesendet und die Funktion startet erneut mit dem Lesen der Startzeit des nächsten Datenpaketes. Die Wiedergabe kann durch die Betätigung des Zurück-Tasters beendet werden. Die Abfrage der Betätigung erfolgt jeweils vor dem Lesen der Startzeit.
% !TEX root = BA-Bauer.tex
\subsubsection{Ausgabe eines DMX-Datenpaketes}
Die Funktion zum Senden von DMX-Datenpakten wurde im Vergleich zu der Praxisarbeit optimiert und vereinfacht. Der Vollständige Sendevorgang wird mit nur zwei Funktionen ausgeführt. Codeausschnitt \ref{code:dmx.c-Transmit} zeigt die Funktion zum Senden eines DMX-Datenpaketes, welche als einzige aufgerufen werden muss um ein Datenpaket zu senden.
\lstinputlisting[
caption = DMX.c: Transmit-Funktion,
label = code:dmx.c-Transmit, 
language = C, 
firstnumber = 66, 
firstline = 66, 
lastline = 79]
{/Users/Felix/Documents/CubeMX/BPA-Code/Core/Src/DMX.c}
Die Übergabeparameter der Funktion sind eine DMX-Struktur ($struct$\footnote{text}) $DMX_TypeDef*\ hdmx$ und die Anzahl der zu übertragenden Datenbytes \textit{uint16\_t size}. In der DMX-Struktur befinden sich Informationen über die zu verwendende UART-Schnittstelle, %Macht nur Sinn wenn das im Code auch benutzt wird
sowie das Array $TxBuffer$ mit den zu übertragenen Datenbytes. Zu beginn der Funktion wird in Zeile 68 der Treiberchip über das Beschreiben des Pins $DMX\_DE\_Pin$ mit einem High-Pegel aktiviert. Daraufhin wird der Ausgangspin der UART-Schnittstelle mithilfe der Funktion $DMX\_set\_TX\_Pin\_manual()$ manuell beschreibbar gemacht und mit einem High-Pegel beschrieben. Das Prinzip dieses Vorgang ist in der Praxisarbeit beschrieben \cite{Bauer2021}. %Seite einfügen
Das Zählerregister ($CNT$) des Timers 11 (\textit{htim11}) wird mit dem Wert 0 überschrieben und das $UIF$-Bit \textit{Update Interrupt Flag} im Statusregister des Timers zurückgesetzt. Dieses Bit wird hardwareseitig gesetzt wenn der Timer übergelaufen ist. Mit der Funktion $SET\_BIT(...)$ wird das $CEN$ Bit gesetzt, wodurch der Timer startet. Der Timer ist zuvor in CubeMX als $one-pulse$-Timer%stimmt das??
konfiguriert. Sobald der Timer übergelaufen ist, wird er gestoppt und zählt den Zähler $CNT$ nicht weiter hoch. Mit der $while$-Abfrage in Zeile 74 wird auf den Überluf des Timers gewartet. Ist er übergelaufen, so wird das $UIF$-Bit wieder zurückgesetzt und der Ausgangspin in Zeile 76 mit einem Low-Pegel beschrieben. Zu diesem Zeitpunkt ist das DMX-Startsignal vollständig ausgegeben und die Datenbytes können gesendet werden. Um der UART-Schnittstelle wieder Zugriff auf den Ausgangspin zu gewähren wird mit der Funktion \textit{DMX\_set\_TX\_Pin\_auto()} die manuelle Beschreibbarkeit des Pins deaktiviert. Anschließend wird mit dem Funktionsaufruf in Zeile 78 die Übertragung der Daten gestartet. Bei dieser Funktion handelt es sich um eine nicht-blockierende Funktion. Die Daten werden Interruptbasiert gesendet. Sind alle Daten vollständig gesendet, so wird durch einen Interrupt der UART-Schnittstelle die Funktion in Codeausschnitt \ref{code:dmx.c-txIT} aufgerufen.
\lstinputlisting[
caption = DMX.c: Interrupt-Funktion ausgehender UART-Daten,
label = code:dmx.c-txIT, 
language = C, 
firstnumber = 120, 
firstline = 120, 
lastline = 128]
{/Users/Felix/Documents/CubeMX/BPA-Code/Core/Src/DMX.c}
Der Inhalt der Funktion soll nur ausgeführt werden, wenn ein aktiver Sendevorgang ausgeführt wird, also wenn die Variable \textit{Univers.sending} den Wert 1 enthält. Ist dies der Fall, so wird der UART-Ausgangspin ein weiteres Mal mit der Funktion \textit{DMX\_set\_TX\_Pin\_manual} manuell beschreibbar gemacht und der Pin mit einem Low-Pegel beschrieben um das $Brake$-Signal des DMX-Protokolls zu senden. Anschließend wird der Zustand der entsprechenden LED invertiert um den Benutzer eine Rückmeldung über einen erfolgreichen Sendevorgang eines Datenpaketes zu geben.\\
\textbf{Optimierungsmöglichkeiten}\\
Wie auch in der Praxisarbeit wird das Signal ohne $Interbytedelay$, also Verzögerungszeit zwischen den einzelnen Datenbytes, gesendet. Dadurch wird das eingehende SIgnal nicht originalgetreu wiedergegeben, jedoch antsteht daraus kein Datenverlust oder eine Veränderung der Wiedergabefrequenz der Datenpakete. Außerdem wird in der aktuellen Version des Programmcodes das $Interbytedelay$ während der Aufnahme nicht gemessen oder registriert. Um diese Funktionalität bei der Wiedergabe zu implementieren muss zunächst eine entsprechende Messung des $Interbytedelays$ während der Aufnahme erfolgen.
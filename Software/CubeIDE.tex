% !TEX root = BA-Bauer.tex

\subsection{STMicroelectronics STM32CubeIDE}
Nachdem der Grundcode von CubeMX generiert ist, wird der Programmcode mithilfe der Entwicklungsumgebung STM32CubeIDE von STMicroelectronics\textsubscript{®} erstellt. Sie ist speziell für die Entwicklung von Programmcode für STM32 Mikrocontroller und -prozessoren ausgelegt und beinhaltet neben der Kompilierung von Code auch debugging-Funktionen. Durch die Einbindung von CubeMX in der Entwicklungsumgebung wird keine weitere Software zum Programmieren eines STM32 Mikrocontrollers oder -prozessors benötigt. Die Konfiguration kann jederzeit in CubeMX geändert und ein neuer Grundcode generiert werden \cite{CubeIDE}. Code, der vom Benutzer geschrieben wurde, wird bei diesem Vorgang nicht verändert. Voraussetzung dafür ist die Position des Benutzer-Programmcodes innerhalb der $USER\ CODE$-Blöcke (\ref{code:usercode}).
\lstinputlisting[
caption = main.c: USER CODE Block,
label = code:usercode, 
language = C, 
firstnumber = 106, 
firstline = 106, 
lastline = 108]
{/Users/Felix/Documents/CubeMX/BPA-Code/Core/Src/main.c}
Code, der zwischen der $BEGIN$- und $END$-Zeile geschrieben steht, wird bei einer Neugenerierung des Grundcodes durch CubeMX beibehalten. Diese Blöcke finden sich an verschiedenen Stellen und Dateien wieder.\\

